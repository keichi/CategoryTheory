%-------------------------------------------------------------------------------
%   Haskellerのための圏論 - はじめに
%   Keichi
%   圏論の勉強成果の個人的なまとめ
%-------------------------------------------------------------------------------

%-------------------------------------------------------------------------------

\title{Haskellerのための圏論}
\author{Keichi}
\maketitle

\section{はじめに}
この文書はHaskellを勉強している筆者が、寄り道して勉強した圏論について、
学習した事項をまとめたものです。主な内容は、HaskellWikiのCategory theoryの
ページ\cite{wiki}の翻訳です。
この文書は、Wikiの内容にHaskellで書いたサンプルコードや、図、用語の解説などを
多少書き加えたものです。

Haskellは、圏論をプログラミング言語に応用してつくられた言語です。
ただし圏論を知らなくても、Haskellの概念を理解することはできますし、
コード自体は書けます。ただしファンクタやモナドなどの概念を考えた側は、
圏論から着想を得ているわけで、圏論を学ぶことでより深くHaskellを理解できると
私は思っています。

