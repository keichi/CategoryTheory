%-------------------------------------------------------------------------------
%   Haskellerのための圏論 - Kleisli圏
%   Keichi
%   圏論の勉強成果の個人的なまとめ
%-------------------------------------------------------------------------------

%-------------------------------------------------------------------------------

\newpage
\section{Kleisli圏}

\subsection{定義}
圏$C$上のKleisli Triple$(T, \eta, (-)^*)$に対して、
以下のように構成することで圏$D$をつくることができる。
このとき、圏DはKleisli圏とよばれる。

\begin{itemize}
    \item $D$の対象は$C$の対象と同じ
    \item $D$の射$f:A\to B$は、$C$の射$f:A\to T(B)$
    \item $D$での射の合成$g\circ f$は,$C$における$g^*\circ f$
    \item $D$での恒等射$id$はCでの$\eta$
\end{itemize}
